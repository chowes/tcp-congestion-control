\section{Background}

Modern distributed cloud applications rely heavily on the \emph{partition/aggregate} design pattern, in which an application is broken in into hierarchical layers and time-sensitive requests at higher layers are divided and delegated to workers in the lower layers. Workers perform some component of a task and return a result to an aggregator, which is combined with results from other workers and passed back up through the hierarchy. A problem arises when workers simultaneously report results back to an aggregator, since this traffic must pass through a shared bottleneck, which results in high queueing delays for time-sensitive traffic. 

% why is this an issue? Discuss incast, partition/aggregate pattern, traffic characterization

\section{Data Center TCP}

Data center TCP (DCTCP) attempts to address the problem of latency in partition/aggregate traffic by reducing queue length without affecting throughput for large TCP flows.

\subsection{Reproducing DCTCP Results}

\subsubsection{Methods}

Selected results from \cite{alizadeh_data_2010} were reproduced using the Mininet network emulator running on Ubuntu 12.04 with a version of the 3.18 Linux kernel patched to add in support for DCTCP. 

\subsubsection{Results}

\subsubsection{Discussion}

\section{TCP BBR}

TCP BBR is a congestion control algorithm based on bottleneck bandwidth and round-trip propogation time.s

\section{Incast TCP}

\section{Multipath TCP}

\section{Conclusions}

